\documentclass[a4paper, 11pt]{article}
\usepackage[utf8x]{inputenc} \usepackage[brazil]{babel}
\usepackage[dvipdfm, a4paper, top=1in, bottom=1in, left=1in, right=1in]{geometry}
\usepackage{hyperref}
\usepackage{multicol}
\usepackage{amsmath}
\usepackage{graphicx}

\begin{document}
\title{Trabalho Pratico II - Compressor de Arquivos}
\author{Joao Paulo Mendes de Sa}
\date{}
\maketitle

\section{Introdução}
Para quantidades enormes de dados os métodos tradicionais de ordenação não são suficientemente adequados. Ordenação em memoria externa é requerida quando todos os dados a serem ordenados não cabem dentro da memoria principal disponível da maquina. Em vez eles tem que permanecer na memoria externa, o hard drive. Para realizar a ordenação desses dados é necessária uma estrategia de ordenar/intercalar onde na primeira etapa os dados são quebrados em pedaços e cada pedaço e ordenado independente dos outros. Em seguida, na segunda etapa, os pedaços ordenados são juntados para formar um único arquivo ordenado. De modo geral o algorítimo é:

\begin{enumerate}
\item Quebra o arquivo em subarquivos que são capazes de caber dentro da memoria disponível.
\item Enquanto não atingir o fim do arquivo:
\begin{enumerate}
\item Carrega na memoria o numero de items que couber.
\item Ordena o conteúdo da memoria.
\item Escreva o conteúdo ordenado para um subarquivo arquivo.
\item Armazene o arquivo ordenado.
\end{enumerate}
\item Reabra todos os arquivos gerados.
\item Carrega na memoria o primeiro item de cada subarquivo.
\item Crie uma fila de prioridades esses items.
\item Enquanto a fila não ficar vazia:
\begin{enumerate}
\item Retire e escreva no arquivo de saída o primeiro item da fila.
\item Adicione a fila o próximo item do subarquivo onde o ultimo retirado originou.
\end{enumerate}
\item Delete todos os subarquivos intermediários gerados.
\end{enumerate}

\section{Implementação}
\subsection{Estrutura de Dados}
Foi criado um tipo abstrato de dados para processar as rodadas gerados, cada rodada composta por um TAD de items que armazenavam cada url e pageviews, e outro para manter a fila de prioridades composta de items.

Para implementar a fila de prioridade foi usado um heap. O TAD usado criava e manipulava um heap para facilitar as operações de remoção e inserção.

\subsection{Funções e Procedimentos}
\begin{verbatim}
void entry_swap(entry_t *a, entry_t *b);
\end{verbatim}
Troca o conteúdo de dois items.

\begin{verbatim}
int round_split(FILE *input, int entryMax);
\end{verbatim}
Quebra o arquivo em múltiplos pedaços. Ordena eles atraves de um quicksort e finalmente escreve o resultado para um arquivo.

\begin{verbatim}
void round_write_file(round_t *round, int roundCur);
\end{verbatim}
Recebe a rodada ordenada e escreve para um arquivo intermediário.

\begin{verbatim}
void round_merge(FILE *output, int roundNum);
\end{verbatim}
Reabre os subarquivos gerados e intercale eles para gerar as saída final ordenada.

\begin{verbatim}
void sort_quick(entry_t* entry, int start, int end);
\end{verbatim}
Função principal do quicksort. Para casos pequenos usa o Shell sort.

\begin{verbatim}
int sort_quick_partition(entry_t* entry, int pivot, int start, int end);
\end{verbatim}
Função de partição do quick sort para dividir e conquistar.

\begin{verbatim}
void queue_fix(entry_t *heap, int heapSize, int father);
\end{verbatim}
Mantem a ordem do heap.

\begin{verbatim}
void queue_build(entry_t *heap, int heapSize);
\end{verbatim}
A partir de um arranjo desorganizado cria um heap.

\begin{verbatim}
entry_t queue_pop(entry_t *heap, int *heapSize);
\end{verbatim}
Retira e retorna o elemento com maior prioridade da fila.

\begin{verbatim}
void queue_push(entry_t *heap, int *heapSize, entry_t insert);
\end{verbatim}
Insere um elemento mantendo a integridade da fila.

\subsection{Programa Principal}
O programa principal é divido em trés etapa. 
\begin{enumerate}
\item Particionar o arquivo de entrada.
\item Ordenar os pedaços.
\item Intercalar os pedaços e produzir a saída. 
\end{enumerate}

Para quebrar os pedaços e lido um trecho do arquivo que cabe na memoria. Esse trecho é analisado para descobrir o url e pageviews de cada item. 

Esses items são ordenados através do quicksort. Finalmente eles são escritos no subarquivo intermediário.

Para intercalar os pedaços, são reabertos todos subarquivos gerados simultaneamente. É lido e analisado o primeiro item de cada subarquivo para construir a fila de prioridades implementada usando um heap. Ate a fila ficar vazio, são escritos para o arquivo final o item com maior prioridade e adicionado a fila o próximo item do subarquivo de onde o item que acabou de sair foi retirado. No fim os subarquivos são removidos.

\subsection{Organização do Código, Decisões de Implementação e Detalhes Técnicos}
O código esta dividido em 3 arquivos: exsort.h contem as declarações does TADs e funções, exsort.c as implementações dos TADs e main.c com o programa principal.
Foi considerado que o tamanho máximo de um url era 100 caracteres para simplificar a implementação e que a ordenação de URLs no caso de empate respeitava a ordem da tabela ASCII. O compilador usado foi GCC 4.6.020110429 no sistema operacional Arch Linux 2.6.39-ARCH. O programa aceita os parametros <arquivo de entrada> <arquivo de saída> <numero máximo de items que cabe na memoria>.

\section{Analise de Complexidade}
A variável $n$ e definida como o numero de items no arquivo de entrada $x$ como o numero de arquivos gerados dependendo da memoria.

\begin{verbatim}
void entry_swap(entry_t *a, entry_t *b);
\end{verbatim}
Simple troca em $O(1)$.

\begin{verbatim}
int round_split(FILE *input, int entryMax);
\end{verbatim}
Para cada $x$, Le o pedaço do arquivo em $O(n)$, e usa sort\_quick() é $O(nlog(n))$, escreve para um subarquivo em $O(n)$. Logo a complexidade é de $O(xnlog(n))$.

\begin{verbatim}
void round_write_file(round_t *roundUnsrt, int roundCur);
\end{verbatim}
Escreve cada item a um arquivo em $O(n)$.

\begin{verbatim}
void round_merge(FILE *output, int roundNum);
\end{verbatim}
Abre os arquivos em $O(x)$, constrói a file em $O(xlog(x))$ por que o numero de items é limitado pelo numero de arquivos gerados. Para cada item $n$ insere $O(log(n))$ e remove $O(log(n))$ que é $O(nlog(n))$. Como o numero de subarquivos nunca e maior que o numero de items totais a função acaba sendo $O(nlog(n))$.

\begin{verbatim}
void sort_quick(entry_t* entry, int start, int end);
\end{verbatim}
Função recursiva por que chara a partição e duas quicksort's para cada metade da entrada com recorrência de $T(n) = 2T(frac{n}{2}) + O(n)$. Resolvendo essa relação $O(nlog(n))$.

\begin{verbatim}
int sort_quick_partition(entry_t* entry, int pivot, int start, int end);
\end{verbatim}
Percorre todos elementos do uma vez logo $O(n)$.

\begin{verbatim}
void queue_fix(entry_t *heap, int heapSize, int father);
\end{verbatim}
Navega um vetor como uma arvore que acaba sendo em $(Olog(n)$.

\begin{verbatim}
void queue_build(entry_t *heap, int heapSize);
\end{verbatim}
Chama queue\_heapify() para metade de n logo $O(nlog(n))$.

\begin{verbatim}
entry_t queue_pop(entry_t *heap, int *heapSize);
\end{verbatim}
Chama queue\_heapify() uma vez então $O(log(n))$.

\begin{verbatim}
void queue_push(entry_t *heap, int *heapSize, entry_t insert);
\end{verbatim}
Como um queue\_heapify() navega como uma a arvore porem de baixo para cima.

\subsection{Programa principal}
A complexidade da implementacao de ordenacao externa acabou sendo:
$$max(O(xnlog(n)), O(nlog(n))) = O(xnlog(n))$$
$x$ depende diretamente do valor da entrada mais vai ser no máximo igual a $n$ no caso onde o programa gera um arquivo por entidade. O que faz a complexidade ser igual a $O(n^2log(n))$

\section{Testes}
Para testar a saida do programa foi desenvolvido um script para gerar arquivos com url e pageviews aleatórios usado como entrada. Usando o sort (UNIX) com parâmetros "-k2nr -k1 test.txt" os testes gerados eram ordenados. Para verificar se a saída do programa estava certa foi usado o diff do vim. Para determinar o tempo de execução para cada arquivo foi feito outro script que rodada 10 vezes um arquivo de mesmo tamanho e pegava a media dos tempos de execução. A curva parece com a de $n^2log(n)$ deslocada por uma constante. 

\begin{figure}[!htb]
\begin{center}
% GNUPLOT: LaTeX picture
\setlength{\unitlength}{0.240900pt}
\ifx\plotpoint\undefined\newsavebox{\plotpoint}\fi
\sbox{\plotpoint}{\rule[-0.200pt]{0.400pt}{0.400pt}}%
\begin{picture}(1500,900)(0,0)
\sbox{\plotpoint}{\rule[-0.200pt]{0.400pt}{0.400pt}}%
\put(151.0,131.0){\rule[-0.200pt]{4.818pt}{0.400pt}}
\put(131,131){\makebox(0,0)[r]{ 0}}
\put(1419.0,131.0){\rule[-0.200pt]{4.818pt}{0.400pt}}
\put(151.0,223.0){\rule[-0.200pt]{4.818pt}{0.400pt}}
\put(131,223){\makebox(0,0)[r]{ 10}}
\put(1419.0,223.0){\rule[-0.200pt]{4.818pt}{0.400pt}}
\put(151.0,315.0){\rule[-0.200pt]{4.818pt}{0.400pt}}
\put(131,315){\makebox(0,0)[r]{ 20}}
\put(1419.0,315.0){\rule[-0.200pt]{4.818pt}{0.400pt}}
\put(151.0,407.0){\rule[-0.200pt]{4.818pt}{0.400pt}}
\put(131,407){\makebox(0,0)[r]{ 30}}
\put(1419.0,407.0){\rule[-0.200pt]{4.818pt}{0.400pt}}
\put(151.0,500.0){\rule[-0.200pt]{4.818pt}{0.400pt}}
\put(131,500){\makebox(0,0)[r]{ 40}}
\put(1419.0,500.0){\rule[-0.200pt]{4.818pt}{0.400pt}}
\put(151.0,592.0){\rule[-0.200pt]{4.818pt}{0.400pt}}
\put(131,592){\makebox(0,0)[r]{ 50}}
\put(1419.0,592.0){\rule[-0.200pt]{4.818pt}{0.400pt}}
\put(151.0,684.0){\rule[-0.200pt]{4.818pt}{0.400pt}}
\put(131,684){\makebox(0,0)[r]{ 60}}
\put(1419.0,684.0){\rule[-0.200pt]{4.818pt}{0.400pt}}
\put(151.0,776.0){\rule[-0.200pt]{4.818pt}{0.400pt}}
\put(131,776){\makebox(0,0)[r]{ 70}}
\put(1419.0,776.0){\rule[-0.200pt]{4.818pt}{0.400pt}}
\put(151.0,131.0){\rule[-0.200pt]{0.400pt}{4.818pt}}
\put(151,90){\makebox(0,0){$10^0$}}
\put(151.0,756.0){\rule[-0.200pt]{0.400pt}{4.818pt}}
\put(206.0,131.0){\rule[-0.200pt]{0.400pt}{2.409pt}}
\put(206.0,766.0){\rule[-0.200pt]{0.400pt}{2.409pt}}
\put(280.0,131.0){\rule[-0.200pt]{0.400pt}{2.409pt}}
\put(280.0,766.0){\rule[-0.200pt]{0.400pt}{2.409pt}}
\put(317.0,131.0){\rule[-0.200pt]{0.400pt}{2.409pt}}
\put(317.0,766.0){\rule[-0.200pt]{0.400pt}{2.409pt}}
\put(335.0,131.0){\rule[-0.200pt]{0.400pt}{4.818pt}}
\put(335,90){\makebox(0,0){$10^1$}}
\put(335.0,756.0){\rule[-0.200pt]{0.400pt}{4.818pt}}
\put(390.0,131.0){\rule[-0.200pt]{0.400pt}{2.409pt}}
\put(390.0,766.0){\rule[-0.200pt]{0.400pt}{2.409pt}}
\put(464.0,131.0){\rule[-0.200pt]{0.400pt}{2.409pt}}
\put(464.0,766.0){\rule[-0.200pt]{0.400pt}{2.409pt}}
\put(501.0,131.0){\rule[-0.200pt]{0.400pt}{2.409pt}}
\put(501.0,766.0){\rule[-0.200pt]{0.400pt}{2.409pt}}
\put(519.0,131.0){\rule[-0.200pt]{0.400pt}{4.818pt}}
\put(519,90){\makebox(0,0){$10^2$}}
\put(519.0,756.0){\rule[-0.200pt]{0.400pt}{4.818pt}}
\put(574.0,131.0){\rule[-0.200pt]{0.400pt}{2.409pt}}
\put(574.0,766.0){\rule[-0.200pt]{0.400pt}{2.409pt}}
\put(648.0,131.0){\rule[-0.200pt]{0.400pt}{2.409pt}}
\put(648.0,766.0){\rule[-0.200pt]{0.400pt}{2.409pt}}
\put(685.0,131.0){\rule[-0.200pt]{0.400pt}{2.409pt}}
\put(685.0,766.0){\rule[-0.200pt]{0.400pt}{2.409pt}}
\put(703.0,131.0){\rule[-0.200pt]{0.400pt}{4.818pt}}
\put(703,90){\makebox(0,0){$10^3$}}
\put(703.0,756.0){\rule[-0.200pt]{0.400pt}{4.818pt}}
\put(758.0,131.0){\rule[-0.200pt]{0.400pt}{2.409pt}}
\put(758.0,766.0){\rule[-0.200pt]{0.400pt}{2.409pt}}
\put(832.0,131.0){\rule[-0.200pt]{0.400pt}{2.409pt}}
\put(832.0,766.0){\rule[-0.200pt]{0.400pt}{2.409pt}}
\put(869.0,131.0){\rule[-0.200pt]{0.400pt}{2.409pt}}
\put(869.0,766.0){\rule[-0.200pt]{0.400pt}{2.409pt}}
\put(887.0,131.0){\rule[-0.200pt]{0.400pt}{4.818pt}}
\put(887,90){\makebox(0,0){$10^4$}}
\put(887.0,756.0){\rule[-0.200pt]{0.400pt}{4.818pt}}
\put(942.0,131.0){\rule[-0.200pt]{0.400pt}{2.409pt}}
\put(942.0,766.0){\rule[-0.200pt]{0.400pt}{2.409pt}}
\put(1016.0,131.0){\rule[-0.200pt]{0.400pt}{2.409pt}}
\put(1016.0,766.0){\rule[-0.200pt]{0.400pt}{2.409pt}}
\put(1053.0,131.0){\rule[-0.200pt]{0.400pt}{2.409pt}}
\put(1053.0,766.0){\rule[-0.200pt]{0.400pt}{2.409pt}}
\put(1071.0,131.0){\rule[-0.200pt]{0.400pt}{4.818pt}}
\put(1071,90){\makebox(0,0){$10^5$}}
\put(1071.0,756.0){\rule[-0.200pt]{0.400pt}{4.818pt}}
\put(1126.0,131.0){\rule[-0.200pt]{0.400pt}{2.409pt}}
\put(1126.0,766.0){\rule[-0.200pt]{0.400pt}{2.409pt}}
\put(1200.0,131.0){\rule[-0.200pt]{0.400pt}{2.409pt}}
\put(1200.0,766.0){\rule[-0.200pt]{0.400pt}{2.409pt}}
\put(1237.0,131.0){\rule[-0.200pt]{0.400pt}{2.409pt}}
\put(1237.0,766.0){\rule[-0.200pt]{0.400pt}{2.409pt}}
\put(1255.0,131.0){\rule[-0.200pt]{0.400pt}{4.818pt}}
\put(1255,90){\makebox(0,0){$10^6$}}
\put(1255.0,756.0){\rule[-0.200pt]{0.400pt}{4.818pt}}
\put(1310.0,131.0){\rule[-0.200pt]{0.400pt}{2.409pt}}
\put(1310.0,766.0){\rule[-0.200pt]{0.400pt}{2.409pt}}
\put(1384.0,131.0){\rule[-0.200pt]{0.400pt}{2.409pt}}
\put(1384.0,766.0){\rule[-0.200pt]{0.400pt}{2.409pt}}
\put(1421.0,131.0){\rule[-0.200pt]{0.400pt}{2.409pt}}
\put(1421.0,766.0){\rule[-0.200pt]{0.400pt}{2.409pt}}
\put(1439.0,131.0){\rule[-0.200pt]{0.400pt}{4.818pt}}
\put(1439,90){\makebox(0,0){$10^7$}}
\put(1439.0,756.0){\rule[-0.200pt]{0.400pt}{4.818pt}}
\put(151.0,131.0){\rule[-0.200pt]{0.400pt}{155.380pt}}
\put(151.0,131.0){\rule[-0.200pt]{310.279pt}{0.400pt}}
\put(1439.0,131.0){\rule[-0.200pt]{0.400pt}{155.380pt}}
\put(151.0,776.0){\rule[-0.200pt]{310.279pt}{0.400pt}}
\put(30,453){\makebox(0,0){\rotatebox{90}{Tempo (s)}}}
\put(795,29){\makebox(0,0){Numero De Entidades}}
\put(795,838){\makebox(0,0){Desempenho Para Arquivos Com Numero De Entidades Diferentes}}
\put(1279,736){\makebox(0,0)[r]{Programa}}
\put(1299.0,736.0){\rule[-0.200pt]{24.090pt}{0.400pt}}
\put(151,131){\usebox{\plotpoint}}
\put(335,130.67){\rule{31.076pt}{0.400pt}}
\multiput(335.00,130.17)(64.500,1.000){2}{\rule{15.538pt}{0.400pt}}
\put(464,130.67){\rule{13.250pt}{0.400pt}}
\multiput(464.00,131.17)(27.500,-1.000){2}{\rule{6.625pt}{0.400pt}}
\put(519,130.67){\rule{31.076pt}{0.400pt}}
\multiput(519.00,130.17)(64.500,1.000){2}{\rule{15.538pt}{0.400pt}}
\put(151.0,131.0){\rule[-0.200pt]{44.326pt}{0.400pt}}
\put(703,131.67){\rule{31.076pt}{0.400pt}}
\multiput(703.00,131.17)(64.500,1.000){2}{\rule{15.538pt}{0.400pt}}
\put(832,132.67){\rule{13.250pt}{0.400pt}}
\multiput(832.00,132.17)(27.500,1.000){2}{\rule{6.625pt}{0.400pt}}
\multiput(887.00,134.58)(5.537,0.492){21}{\rule{4.400pt}{0.119pt}}
\multiput(887.00,133.17)(119.868,12.000){2}{\rule{2.200pt}{0.400pt}}
\multiput(1016.00,146.58)(1.865,0.494){27}{\rule{1.567pt}{0.119pt}}
\multiput(1016.00,145.17)(51.748,15.000){2}{\rule{0.783pt}{0.400pt}}
\multiput(1071.00,161.58)(0.561,0.499){227}{\rule{0.549pt}{0.120pt}}
\multiput(1071.00,160.17)(127.861,115.000){2}{\rule{0.274pt}{0.400pt}}
\multiput(1200.58,276.00)(0.499,1.533){107}{\rule{0.120pt}{1.322pt}}
\multiput(1199.17,276.00)(55.000,165.257){2}{\rule{0.400pt}{0.661pt}}
\multiput(1255.58,444.00)(0.499,3.015){107}{\rule{0.120pt}{2.500pt}}
\multiput(1254.17,444.00)(55.000,324.811){2}{\rule{0.400pt}{1.250pt}}
\put(151,131){\makebox(0,0){$\bullet$}}
\put(280,131){\makebox(0,0){$\bullet$}}
\put(335,131){\makebox(0,0){$\bullet$}}
\put(464,132){\makebox(0,0){$\bullet$}}
\put(519,131){\makebox(0,0){$\bullet$}}
\put(648,132){\makebox(0,0){$\bullet$}}
\put(703,132){\makebox(0,0){$\bullet$}}
\put(832,133){\makebox(0,0){$\bullet$}}
\put(887,134){\makebox(0,0){$\bullet$}}
\put(1016,146){\makebox(0,0){$\bullet$}}
\put(1071,161){\makebox(0,0){$\bullet$}}
\put(1200,276){\makebox(0,0){$\bullet$}}
\put(1255,444){\makebox(0,0){$\bullet$}}
\put(1310,774){\makebox(0,0){$\bullet$}}
\put(1349,736){\makebox(0,0){$\bullet$}}
\put(648.0,132.0){\rule[-0.200pt]{13.249pt}{0.400pt}}
\put(1279,695){\makebox(0,0)[r]{$n^2log(n)$}}
\put(1299.0,695.0){\rule[-0.200pt]{24.090pt}{0.400pt}}
\put(151,131){\usebox{\plotpoint}}
\put(151,131.17){\rule{2.500pt}{0.400pt}}
\multiput(151.00,130.17)(6.811,2.000){2}{\rule{1.250pt}{0.400pt}}
\multiput(163.00,133.61)(2.248,0.447){3}{\rule{1.567pt}{0.108pt}}
\multiput(163.00,132.17)(7.748,3.000){2}{\rule{0.783pt}{0.400pt}}
\multiput(174.00,136.59)(1.267,0.477){7}{\rule{1.060pt}{0.115pt}}
\multiput(174.00,135.17)(9.800,5.000){2}{\rule{0.530pt}{0.400pt}}
\multiput(186.00,141.59)(0.874,0.485){11}{\rule{0.786pt}{0.117pt}}
\multiput(186.00,140.17)(10.369,7.000){2}{\rule{0.393pt}{0.400pt}}
\multiput(198.00,148.58)(0.496,0.492){21}{\rule{0.500pt}{0.119pt}}
\multiput(198.00,147.17)(10.962,12.000){2}{\rule{0.250pt}{0.400pt}}
\multiput(210.58,160.00)(0.492,0.826){19}{\rule{0.118pt}{0.755pt}}
\multiput(209.17,160.00)(11.000,16.434){2}{\rule{0.400pt}{0.377pt}}
\multiput(221.58,178.00)(0.492,1.142){21}{\rule{0.119pt}{1.000pt}}
\multiput(220.17,178.00)(12.000,24.924){2}{\rule{0.400pt}{0.500pt}}
\multiput(233.58,205.00)(0.492,1.659){21}{\rule{0.119pt}{1.400pt}}
\multiput(232.17,205.00)(12.000,36.094){2}{\rule{0.400pt}{0.700pt}}
\multiput(245.58,244.00)(0.492,2.666){19}{\rule{0.118pt}{2.173pt}}
\multiput(244.17,244.00)(11.000,52.490){2}{\rule{0.400pt}{1.086pt}}
\multiput(256.58,301.00)(0.492,3.555){21}{\rule{0.119pt}{2.867pt}}
\multiput(255.17,301.00)(12.000,77.050){2}{\rule{0.400pt}{1.433pt}}
\multiput(268.58,384.00)(0.492,5.149){21}{\rule{0.119pt}{4.100pt}}
\multiput(267.17,384.00)(12.000,111.490){2}{\rule{0.400pt}{2.050pt}}
\multiput(280.58,504.00)(0.492,7.432){21}{\rule{0.119pt}{5.867pt}}
\multiput(279.17,504.00)(12.000,160.823){2}{\rule{0.400pt}{2.933pt}}
\multiput(292.60,677.00)(0.468,14.372){5}{\rule{0.113pt}{10.000pt}}
\multiput(291.17,677.00)(4.000,78.244){2}{\rule{0.400pt}{5.000pt}}
\put(151.0,131.0){\rule[-0.200pt]{0.400pt}{155.380pt}}
\put(151.0,131.0){\rule[-0.200pt]{310.279pt}{0.400pt}}
\put(1439.0,131.0){\rule[-0.200pt]{0.400pt}{155.380pt}}
\put(151.0,776.0){\rule[-0.200pt]{310.279pt}{0.400pt}}
\end{picture}

\end{center}
\caption{Tempo de execução para vários arquivos}
\end{figure}

Foram feitos também, testes para determinar o impacto do limite de memoria. O tempo de execução parece estar mais ou menos constate. Minha hipótese por isso e que as operações em disco são as que gastam mais tempo. Como independente do limite de entidades que cabem na memoria terão que ser feitas $n$ leituras ao arquivo o tempo de execução não muda muito.

\begin{figure}[!htb]
\begin{center}
% GNUPLOT: LaTeX picture
\setlength{\unitlength}{0.240900pt}
\ifx\plotpoint\undefined\newsavebox{\plotpoint}\fi
\sbox{\plotpoint}{\rule[-0.200pt]{0.400pt}{0.400pt}}%
\begin{picture}(1500,900)(0,0)
\sbox{\plotpoint}{\rule[-0.200pt]{0.400pt}{0.400pt}}%
\put(151.0,131.0){\rule[-0.200pt]{4.818pt}{0.400pt}}
\put(131,131){\makebox(0,0)[r]{ 0}}
\put(1419.0,131.0){\rule[-0.200pt]{4.818pt}{0.400pt}}
\put(151.0,223.0){\rule[-0.200pt]{4.818pt}{0.400pt}}
\put(131,223){\makebox(0,0)[r]{ 10}}
\put(1419.0,223.0){\rule[-0.200pt]{4.818pt}{0.400pt}}
\put(151.0,315.0){\rule[-0.200pt]{4.818pt}{0.400pt}}
\put(131,315){\makebox(0,0)[r]{ 20}}
\put(1419.0,315.0){\rule[-0.200pt]{4.818pt}{0.400pt}}
\put(151.0,407.0){\rule[-0.200pt]{4.818pt}{0.400pt}}
\put(131,407){\makebox(0,0)[r]{ 30}}
\put(1419.0,407.0){\rule[-0.200pt]{4.818pt}{0.400pt}}
\put(151.0,500.0){\rule[-0.200pt]{4.818pt}{0.400pt}}
\put(131,500){\makebox(0,0)[r]{ 40}}
\put(1419.0,500.0){\rule[-0.200pt]{4.818pt}{0.400pt}}
\put(151.0,592.0){\rule[-0.200pt]{4.818pt}{0.400pt}}
\put(131,592){\makebox(0,0)[r]{ 50}}
\put(1419.0,592.0){\rule[-0.200pt]{4.818pt}{0.400pt}}
\put(151.0,684.0){\rule[-0.200pt]{4.818pt}{0.400pt}}
\put(131,684){\makebox(0,0)[r]{ 60}}
\put(1419.0,684.0){\rule[-0.200pt]{4.818pt}{0.400pt}}
\put(151.0,776.0){\rule[-0.200pt]{4.818pt}{0.400pt}}
\put(131,776){\makebox(0,0)[r]{ 70}}
\put(1419.0,776.0){\rule[-0.200pt]{4.818pt}{0.400pt}}
\put(151.0,131.0){\rule[-0.200pt]{0.400pt}{4.818pt}}
\put(151,90){\makebox(0,0){$10^3$}}
\put(151.0,756.0){\rule[-0.200pt]{0.400pt}{4.818pt}}
\put(280.0,131.0){\rule[-0.200pt]{0.400pt}{2.409pt}}
\put(280.0,766.0){\rule[-0.200pt]{0.400pt}{2.409pt}}
\put(356.0,131.0){\rule[-0.200pt]{0.400pt}{2.409pt}}
\put(356.0,766.0){\rule[-0.200pt]{0.400pt}{2.409pt}}
\put(409.0,131.0){\rule[-0.200pt]{0.400pt}{2.409pt}}
\put(409.0,766.0){\rule[-0.200pt]{0.400pt}{2.409pt}}
\put(451.0,131.0){\rule[-0.200pt]{0.400pt}{2.409pt}}
\put(451.0,766.0){\rule[-0.200pt]{0.400pt}{2.409pt}}
\put(485.0,131.0){\rule[-0.200pt]{0.400pt}{2.409pt}}
\put(485.0,766.0){\rule[-0.200pt]{0.400pt}{2.409pt}}
\put(514.0,131.0){\rule[-0.200pt]{0.400pt}{2.409pt}}
\put(514.0,766.0){\rule[-0.200pt]{0.400pt}{2.409pt}}
\put(539.0,131.0){\rule[-0.200pt]{0.400pt}{2.409pt}}
\put(539.0,766.0){\rule[-0.200pt]{0.400pt}{2.409pt}}
\put(561.0,131.0){\rule[-0.200pt]{0.400pt}{2.409pt}}
\put(561.0,766.0){\rule[-0.200pt]{0.400pt}{2.409pt}}
\put(580.0,131.0){\rule[-0.200pt]{0.400pt}{4.818pt}}
\put(580,90){\makebox(0,0){$10^4$}}
\put(580.0,756.0){\rule[-0.200pt]{0.400pt}{4.818pt}}
\put(710.0,131.0){\rule[-0.200pt]{0.400pt}{2.409pt}}
\put(710.0,766.0){\rule[-0.200pt]{0.400pt}{2.409pt}}
\put(785.0,131.0){\rule[-0.200pt]{0.400pt}{2.409pt}}
\put(785.0,766.0){\rule[-0.200pt]{0.400pt}{2.409pt}}
\put(839.0,131.0){\rule[-0.200pt]{0.400pt}{2.409pt}}
\put(839.0,766.0){\rule[-0.200pt]{0.400pt}{2.409pt}}
\put(880.0,131.0){\rule[-0.200pt]{0.400pt}{2.409pt}}
\put(880.0,766.0){\rule[-0.200pt]{0.400pt}{2.409pt}}
\put(914.0,131.0){\rule[-0.200pt]{0.400pt}{2.409pt}}
\put(914.0,766.0){\rule[-0.200pt]{0.400pt}{2.409pt}}
\put(943.0,131.0){\rule[-0.200pt]{0.400pt}{2.409pt}}
\put(943.0,766.0){\rule[-0.200pt]{0.400pt}{2.409pt}}
\put(968.0,131.0){\rule[-0.200pt]{0.400pt}{2.409pt}}
\put(968.0,766.0){\rule[-0.200pt]{0.400pt}{2.409pt}}
\put(990.0,131.0){\rule[-0.200pt]{0.400pt}{2.409pt}}
\put(990.0,766.0){\rule[-0.200pt]{0.400pt}{2.409pt}}
\put(1010.0,131.0){\rule[-0.200pt]{0.400pt}{4.818pt}}
\put(1010,90){\makebox(0,0){$10^5$}}
\put(1010.0,756.0){\rule[-0.200pt]{0.400pt}{4.818pt}}
\put(1139.0,131.0){\rule[-0.200pt]{0.400pt}{2.409pt}}
\put(1139.0,766.0){\rule[-0.200pt]{0.400pt}{2.409pt}}
\put(1215.0,131.0){\rule[-0.200pt]{0.400pt}{2.409pt}}
\put(1215.0,766.0){\rule[-0.200pt]{0.400pt}{2.409pt}}
\put(1268.0,131.0){\rule[-0.200pt]{0.400pt}{2.409pt}}
\put(1268.0,766.0){\rule[-0.200pt]{0.400pt}{2.409pt}}
\put(1310.0,131.0){\rule[-0.200pt]{0.400pt}{2.409pt}}
\put(1310.0,766.0){\rule[-0.200pt]{0.400pt}{2.409pt}}
\put(1344.0,131.0){\rule[-0.200pt]{0.400pt}{2.409pt}}
\put(1344.0,766.0){\rule[-0.200pt]{0.400pt}{2.409pt}}
\put(1372.0,131.0){\rule[-0.200pt]{0.400pt}{2.409pt}}
\put(1372.0,766.0){\rule[-0.200pt]{0.400pt}{2.409pt}}
\put(1397.0,131.0){\rule[-0.200pt]{0.400pt}{2.409pt}}
\put(1397.0,766.0){\rule[-0.200pt]{0.400pt}{2.409pt}}
\put(1419.0,131.0){\rule[-0.200pt]{0.400pt}{2.409pt}}
\put(1419.0,766.0){\rule[-0.200pt]{0.400pt}{2.409pt}}
\put(1439.0,131.0){\rule[-0.200pt]{0.400pt}{4.818pt}}
\put(1439,90){\makebox(0,0){$10^6$}}
\put(1439.0,756.0){\rule[-0.200pt]{0.400pt}{4.818pt}}
\put(151.0,131.0){\rule[-0.200pt]{0.400pt}{155.380pt}}
\put(151.0,131.0){\rule[-0.200pt]{310.279pt}{0.400pt}}
\put(1439.0,131.0){\rule[-0.200pt]{0.400pt}{155.380pt}}
\put(151.0,776.0){\rule[-0.200pt]{310.279pt}{0.400pt}}
\put(30,453){\makebox(0,0){\rotatebox{90}{Tempo (s)}}}
\put(795,29){\makebox(0,0){Limite De Entidades}}
\put(795,838){\makebox(0,0){Desempenho Para Limite De Entidades Diferentes}}
\put(1439,442){\usebox{\plotpoint}}
\multiput(1416.17,440.94)(-7.792,-0.468){5}{\rule{5.500pt}{0.113pt}}
\multiput(1427.58,441.17)(-42.584,-4.000){2}{\rule{2.750pt}{0.400pt}}
\multiput(1374.21,438.58)(-3.210,0.492){21}{\rule{2.600pt}{0.119pt}}
\multiput(1379.60,437.17)(-69.604,12.000){2}{\rule{1.300pt}{0.400pt}}
\multiput(1293.11,448.92)(-5.096,-0.493){23}{\rule{4.069pt}{0.119pt}}
\multiput(1301.55,449.17)(-120.554,-13.000){2}{\rule{2.035pt}{0.400pt}}
\multiput(1109.60,435.94)(-24.900,-0.468){5}{\rule{17.200pt}{0.113pt}}
\multiput(1145.30,436.17)(-135.301,-4.000){2}{\rule{8.600pt}{0.400pt}}
\multiput(973.61,431.93)(-11.701,-0.482){9}{\rule{8.767pt}{0.116pt}}
\multiput(991.80,432.17)(-111.804,-6.000){2}{\rule{4.383pt}{0.400pt}}
\multiput(808.19,425.95)(-28.593,-0.447){3}{\rule{17.300pt}{0.108pt}}
\multiput(844.09,426.17)(-93.093,-3.000){2}{\rule{8.650pt}{0.400pt}}
\multiput(724.77,424.58)(-8.043,0.492){19}{\rule{6.318pt}{0.118pt}}
\multiput(737.89,423.17)(-157.886,11.000){2}{\rule{3.159pt}{0.400pt}}
\multiput(536.75,433.93)(-14.291,-0.477){7}{\rule{10.420pt}{0.115pt}}
\multiput(558.37,434.17)(-107.373,-5.000){2}{\rule{5.210pt}{0.400pt}}
\multiput(433.41,430.58)(-5.230,0.497){55}{\rule{4.238pt}{0.120pt}}
\multiput(442.20,429.17)(-291.204,29.000){2}{\rule{2.119pt}{0.400pt}}
\put(1439,442){\makebox(0,0){$\bullet$}}
\put(1385,438){\makebox(0,0){$\bullet$}}
\put(1310,450){\makebox(0,0){$\bullet$}}
\put(1181,437){\makebox(0,0){$\bullet$}}
\put(1010,433){\makebox(0,0){$\bullet$}}
\put(880,427){\makebox(0,0){$\bullet$}}
\put(751,424){\makebox(0,0){$\bullet$}}
\put(580,435){\makebox(0,0){$\bullet$}}
\put(451,430){\makebox(0,0){$\bullet$}}
\put(151,459){\makebox(0,0){$\bullet$}}
\put(151.0,131.0){\rule[-0.200pt]{0.400pt}{155.380pt}}
\put(151.0,131.0){\rule[-0.200pt]{310.279pt}{0.400pt}}
\put(1439.0,131.0){\rule[-0.200pt]{0.400pt}{155.380pt}}
\put(151.0,776.0){\rule[-0.200pt]{310.279pt}{0.400pt}}
\end{picture}

\end{center}
\caption{Efeito do limite de entidade}
\end{figure}

\section{Conclusão}
Eu percebi que o maior gargalo na implementação era as operações que envolviam o disco. Eu pensei em algumas maneiras de melhorar essa porcão do código. Pensei em reduzir a quantidade de leitura ao disco no inicio pegando a maior string que coubesse na memoria e processando ele de uma vez ao invés de ler cada item um por um. Como o limite de URL era 100 caracteres assumi que a memoria máxima dada por esse limite vezes o numero de entidades dado na entrada. Uma consequência disso foi que o numero de items em cada subarquivo variaria para preencher o máximo possível. Na hora do quicksort isso ajudaria por que não seria necessário    fazer varias leituras de arquivos pequenos e a ordenação poderia acontecer diretamente na memoria reduzindo o acesso ao disco. 

Para o quicksort em si, quando ele particiona casos pequenos eu tentei usar o shell sort ao invés de mais chamadas do quicksort que para ter um melhor desempenho em vetores menores, porem em vários testes esse não foi o caso que me levou a retirar esse trecho. Não consegui usar o mesmo método na hora de intercalar pois e necessário ler o topo de cada subarquivo. Neste caso tive que ler cada entrada uma por uma.
\section{Referencias}
\begin{itemize} 
\item Cormen T., Leiserson C., Rivest R., Stein C., Introduction to Algorithms 3rd Edition 2009.
\item \url{http://en.wikipedia.org/External\_sorting}
\end{itemize}
\end{document}
